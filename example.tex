\documentclass[12pt,parskip]{scrartcl}

%%% babel
\usepackage{libertine}
\usepackage[english]{babel}
\usepackage{babel-iast/babel-iast}
\babelfont[iast]{rm}[Renderer=Harfbuzz]{Sanskrit 2003}


%%% ekdosis
\usepackage[teiexport=tidy,parnotes=true]{ekdosis}
\SetLineation{lineation=page,modulo}
\renewcommand{\linenumberfont}{\selectlanguage{english}\footnotesize}

\SetTEIxmlExport{autopar=false}
\SetHooks{refnumstyle=\bfseries,}

\DeclareApparatus{default}[
delim=\hskip0.75em,
bhook= \selectlanguage{english},
ehook=.]

\DeclareApparatus{philcomm}[
delim=\\,
sep={: },
bhook=\selectlanguage{english} \textbf{Philological commentary:}\\,
ehook=.]

% List of Witnesses
\DeclareWitness{R}{R}{Raghunātha Temple Library 4383}[settlement=Jammu]
\DeclareWitness{V}{V}{Vulgate, i.e. Brahmānanda's version}[]

% List of Scholars
\DeclareScholar{edi}{editor}[
forename=Ed,
surname=Itor]

\usepackage{xparse}

%%% define environments and commands
\NewDocumentEnvironment{tlg}{O{}O{}}{\begin{verse}}{\hfill #1\\ \end{verse}}
\NewDocumentCommand{\tl}{m}{{\selectlanguage{iast} #1}}

\NewDocumentEnvironment{cjyo}{O{}}{\begin{center}[\emph{jyotsnā}-commentary to four-chapter-version]\end{center} \begin{otherlanguage}{iast}}{\end{otherlanguage}}
\NewDocumentEnvironment{cpra}{O{}}{\begin{center}[\emph{prakāśikā}-commentary to ten-chapter-version]\end{center} \begin{otherlanguage}{iast}}{\end{otherlanguage}}

%%% modify environments and commands


%%% TEI mapping
\TeXtoTEIPat{\begin {tlg}[#1][#2]}{<lg xml:id="#1" corresp="#tenchap_#2">}
  \TeXtoTEIPat{\end {tlg}}{</lg>}

\TeXtoTEIPat{\begin {cjyo}[#1]}{<p xml:id="jyotsnA_#1">}
  \TeXtoTEIPat{\end {cjyo}}{</p>}

\TeXtoTEIPat{\begin {cpra}[#1]}{<p xml:id="prakAzikA_#1">}
  \TeXtoTEIPat{\end {cpra}}{</p>}

\TeXtoTEIPat{\\}{}

\TeXtoTEI{tl}{l}
\TeXtoTEI{emph}{hi}

\author{Svātmārāma}
\title{Haṭhayogapradīpikā}
\date{}

\begin{document}

\maketitle

\begin{ekdosis}
  \begin{tlg}[4.1][7.1]
    \note[type=philcomm, labelb=l4.1b, labele=l4.1e, lem={4.1}]{The existence of a \emph{maṅgalaśloka} here in what appears to be the middle of the text is not going unnoted by the two commentators (\emph{śiṣṭācāraprāptaṃ maṅgalaṃ granthamadhye racayati} […]). Since the oldest available manuscript from Vārāṇāsī has a only three chapters, this might indicate that the fourth chapter was an addition.}
    \tl{namaḥ śivāya gurave nādabindukalātmane /}\\
    \tl{nirañjanapadaṃ yāti nityaṃ yatra parāyaṇaḥ //}\linelabel{l4.1e}
  \end{tlg}

  \begin{cjyo}[4.1]
    prathamadvitīyatṛtīyopadeśoktānām-āsanakumbhakamudrāṇāṃ phalabhūtaṃ rājayogaṃ vivakṣuḥ svātmārāmaḥ śreyāṃsi bahuvighnānīti tatra vighnabāhulyasya sambhavāt-tannivṛttaye śivābhinnagurunamaskārātmakaṃ maṅgalam-ācarati nama iti /
    śivāya sukharūpeśvarābhinnāya vā /
    tad-uktaṃ namas-te nātha bhagavan-śivāya gururūpiṇe iti /
    gurave deśikāya /
    yad-vā, gurave sarvāntaryāmitayā nikhilopadeṣṭre śiveśvarāya /
    tathā ca pātañjalasūtram-pūrveṣām-api guruḥ kālenānavacchedāt [yo.sū. 1.26] iti /
    namaḥ prahvībhāvo 'stu /
    
    kīdṛśāya śivāya ? gurave nādabindukalātmane /
    kāṃsyaghaṇṭānirhrādavad-anuraṇanaṃ nādaḥ /
    bindur-anusvārottarabhāvī dhvaniḥ /
    kalā nādaikadeśaḥ /
    tā ātmā svarūpaṃ yasya sa tathā tasmai /
    nādabindukalātmanā vartamānāyety-arthaḥ /
    tatra nādabindukalātmani śive gurau nityaṃ pratidinaṃ parāyaṇo 'vahitaḥ pumān /
    etena nādānusandhānaparāyaṇa ity-uktam /
    pūrvapādena guruśivayor-abhedaś-ca sūcitaḥ /
    añjanaṃ māyopādhis-tadrahitaṃ nirañjanaṃ śuddham, padyate gamyate yogibhir-iti padaṃ brahma yāti prāpnoti /
    tathā ca vakṣyati, nādānusandhānasamādhibhājām [4.81] ity ādinā //1//
  \end{cjyo}

  \begin{cpra}[7.1]
    dhyāna …
  \end{cpra}

  \begin{tlg}[4.2][7.2]
    \tl{athedānīṃ pravakṣyāmi samādhikramam \app{
        \lem[wit={V}]{uttamam}
        \rdg[wit={R}]{lakṣaṇam}} /}\\
    \tl{mṛtyughnaṃ ca sukhopāyaṃ brahmānandakaraṃ param //}
  \end{tlg}

  \begin{cjyo}[4.2]
    samādhikramaṃ pratijānīte: atheti /
    athāsanakumbhakamudrākathanānataram-idānīm-asminn-avasare samādhikramam pratyāhārādirūpaṃ pravakṣyāmi prakarṣeṇa vivicya vakṣyāmīty-anvayaḥ /
    kīdṛśaṃ samādhikramam ? uttamam śrīādināthokasapādakoṭisamādhiprakāreṣūtkṛṣṭam /
    punaḥ kīdṛśam ? mṛtyuṃ kālaṃ hanti nivārayatīti mṛtyughnaṃ svecchayā dehatyāgajanakaṃ tattvajñānodayamanonāśavāsanākṣayaiḥ sukhasya jīvanmuktisukhasyopāyaṃ prāptisādhanam /
    punaḥ kīdṛśam ? brahmānandakaraṃ paraṃ brahmānandakaraṃ prārabdhakarmakṣaye sati jīvabrahmaṇor-abhedenātyantikabrahmānandaprāptirūpavidehamuktikaram /
    tatra nirodhasamādhinā cittasya sasaṃskārāśeṣavṛttinirodhe śāntaghoramūḍhāvasthānivṛttau, jīvann-eva hi vidvān-harṣaśokābhyāṃ vimucyate ity ādiśrutyuktanirvikārasvarūpāvasthitirūpā jīvanmuktir-bhavati /

    paramamuktis-tu prārabdhabhogānte 'ntaḥkaraṇaguṇānāṃ pratiprasavenaupādhikarūpātyantikanivṛttāv-ātyantikasvarūpāvasthānaṃ pratiprasavasiddham /
    vyutthānanirodhasamādhisaṃskārā manasi līyante /
    mano'smitāyām asmitā mahati mahān-pradhān-iti cittaguṇānāṃ pratiprasavaḥ pratisargaḥ svakāraṇe layaḥ /

    nanu jīvanmuktasya vyutthāne brāhmaṇo 'haṃ manuṣyo 'ham ity-ādi vyavahāradarśanāc-cittādibhir-aupādhikabhāvajananād amlena dugdhasyeva svarūpacyutiḥ syād iti cen-na /
    samprajñātasamādhau anubhūtātmasaṃskārasya nirodhasaṃskārasya ca tadānīṃ sattvāt /
    tābhyāṃ ca vyutthānasaṃskārasya dagdhabījakalpatvād-vyutthānavyavahārasyātāttvikatvaniścayāt /
    atāttvikānyathābhāvasya vikāritvāprayojakatvāt /
    amlena dugdhasya dadhibhāvas-tu tāttvika iti dṛṣṭāntavaiṣamyāc-ca /
    puruṣasya tv-antaḥkaraṇopādhiko 'haṃ brāhmaṇa ity-ādivyavahāraḥ sphaṭikasya japākusumasaṃnidhānopādhir-aruṇim-eva na tāttvikaḥ /
    japākusumāpagame sphaṭikasya svasvarūpasthititvad-antaḥkaraṇasya sakalavṛttinirodhe svasvarūpāvasthitir-acyutaiva puruṣasya //2//
  \end{cjyo}

  \begin{cpra}[7.2]
    samādhibhedān …
  \end{cpra}

  \begin{tlg}[4.3][8.50]\note[type=philcomm, labelb=l4.3b, labele=l4.3e, lem={4.3}]{In the ten-chapter version the two verses containing synonyms of rājayoga comes at the end of chapter 8. Despite transmitting a clearly related passage this version has some important readings.}%
    \tl{rājayogaḥ samādhiś-ca unmanī ca manonmanī /}\\
    \tl{amaratvaṃ layas-tattvaṃ śūnyāśūnyaṃ paraṃ padam //}\linelabel{l4.3e}
  \end{tlg}

  \begin{cjyo}[4.3]
    samādhiparyāyān-viśeṣeṇāha rājayoga ity-ādinā ślokadvayena /
    spaṣṭam //3-4//
  \end{cjyo}

  \begin{cpra}[8.50]
    rājayogaprayāyān-āha dvābhyām /
  \end{cpra}
  
  \begin{tlg}[4.4][8.51]
    \tl{amanaskaṃ tathādvaitaṃ nirālambaṃ nirañjanam /}\\
    \tl{jīvanmuktiś-ca sahajā turyā cety ekavācakāḥ /}
  \end{tlg}

  \begin{tlg}[4.5][7.4]
    \tl{salile saindhavaṃ yadvat-sāmyaṃ bhajati yogataḥ /}\\
    \tl{tathātmamanasor-aikyaṃ samādhir-abhidhīyate //}
  \end{tlg}

  \begin{cjyo}[4.5]
    tribhiḥ samādhim āha: salila iti /
    yadvat yathā saindhavaṃ sindhudeśodbhavaṃ lavaṇaṃ salile jale yogataḥ saṃyogāt sāmyaṃ salilasāmyaṃ salilaikyaṃ bhajati prāpnoti, tathā tadvad ātmā ca manaś-cātmamanasī tayor-ātmamanasor-aikyam-ekākāratā /
    ātmani dhāritaṃ mana ātmākāraṃ sadātmasāmyaṃ bhajati, tādṛśam-ātmamanasor-aikyaṃ samādhir-abhidhīyate samādhiśabdenocyate ity-arthaḥ //5//
  \end{cjyo}

  \begin{cpra}[8.51]
    samādhyantaram-āha ambusaindhavayor-iti /
  \end{cpra}
  
\end{ekdosis}
\end{document}