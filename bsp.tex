\documentclass[12pt]{article} 
\usepackage[english]{babel}
\usepackage{babel-iast}
\babelfont[iast]{rm}[Renderer=Harfbuzz]{AdishilaSan}
\babelfont[english]{rm}{Adobe Text Pro}
\usepackage[teiexport=tidy,parnotes=roman,verse]{ekdosis}
\SetLineation{lineation=page,modulo}
\SetApparatus{
  bhook = \selectlanguage{english},
  sep = {] }
}

%\AtBeginEnvironment{ekdverse}{\selectlanguage{iast}}
%\EnvtoTEI{ekdverse}{lg}[xml:lang="sa"]

% Macros und Definitionen für den Druck der Siglen
\def\acpc#1#2#3{{#1}\rlap{\textrm{\textsuperscript{#3}}}\textsubscript{\textrm{#2}}\space}
\def\sigl#1#2{{{#1}}\textsubscript{\textrm{#2}}}
\def\Jone{{\sigl{J}{1}}} \def\Joneac{\acpc{J}{1}{ac}\,} \def\Jonepc{\acpc{J}{1}{pc}\,}
\def\Jtwo{{\sigl{J}{2}}} \def\Jtwoac{\acpc{J}{2}{ac}\,} \def\Jtwopc{\acpc{J}{2}{pc}\,}
\def\Jthree{{\sigl{J}{3}}} \def\Jthreeac{\acpc{J}{3}{ac}\,} \def\Jthreepc{\acpc{J}{3}{pc}\,}
\def\Jfour{{\sigl{J}{4}}} \def\Jfourac{\acpc{J}{4}{ac}\,} \def\Jfourpc{\acpc{J}{4}{pc}\,}                 

%  List of Witnesses
\DeclareWitness{S}{S}{Sampūrṇānanda 30109}[settlement=Vārāṇāsī]     % some pṛṣṭhamātra vovels
\DeclareWitness{R}{R}{Raghunātha Temple Library 4383}[settlement=Jammu]  
\DeclareWitness{V}{V}{Vulgate, i.e. Brahmānanda's version}[]       
\DeclareWitness{J1}{\Jone}{Jodhpur 02231}[]
\DeclareWitness{J2}{\Jtwo}{Jodhpur 02232}[]  % too many simple writing mistakes  not collated further: 
\DeclareWitness{J3}{\Jthree}{Jodhpur 02233}[]  % 4 chapters, 93 jpgs,  very readable, but many writing mistakes
\DeclareWitness{J4}{\Jfour}{Jodhpur 02234}[]  % 4 chapters, 85 jpgs,  fewer writing mistakes, viśvanāthena likhitam 
\DeclareWitness{J5}{J5}{Jodhpur 02235}[]  % 4 chapters, 34 jpgs,   long colophon, missing lines in the beginning.
\DeclareWitness{J6}{J6}{Jodhpur 02237}[]  % 4 chapters, 49 jpgs,   1st folio: idaṃ gulābarāyasya tulasīrāmaśarmmaṇaḥ putrasya pustakaṃ ...    End: iti śrīsahajānandasantānacintāmaṇisvātmārāmaviracitāyāṃ ..    saṃvat 1802   (more consistent text)
\DeclareWitness{J7}{J7}{Jodhpur 02241}[]  % 4 chapters, 41 jpgs
\DeclareWitness{J8}{J8}{Jodhpur 23709}[]  % 4 chapters,  87 jpgs.   saṃvat 1724
\DeclareWitness{J9}{J9}{Jodhpur 02224}[]  %  fragment, 20 jpgs. 
%  Haṭhapradīpikā with (non-Sanskrit) Bhāṣya RORI Jodhpur ACC.NO.18552 
%  Haṭhapradīpikā with (non-Sanskrit) commentary, RORI Alwar 952, 4 chapters,  colophon of the comm: iti śrīlāhorīmiśravrajabhūṣanaviracitāyāṃ bhāvārthadīpikāyāṃ caturthodhyāya ..    
%  Haṭhapradīpikā (5 chapter) MSPP Jodhpur ACC.NO.02229/
%  Haṭhapradīpikā (10 chapter with the Prakāśikā MSPP JodhpurACC.NO.02228:     Sanskrit commentary!!

\parindent=3pt

\begin{document}
\begin{otherlanguage}{iast}
\begin{ekdosis}
\begin{ekdverse}        

%%%  Haupttext (=4-chapter version mit Kommentar Jyotsnā)  
namaḥ śivāya gurave nādabindukalātmane  |
nirañjanapadaṃ yāti nityaṃ yatra parāyaṇaḥ  || 4.1 ||

%%% Hier kämen noch Varianten dazu.

%%%%% Mit —–Category habe ich Zusatzinfos bezeichnet, als das, was man sonst noch gerne am Rand einblenden würde:
% – Konkordanz mit anderen Versionen, hier als Beispiel nur die mit der 10-Kapitel-Version
% – Kommentar Jyotsnā (zur 4-Kap.-Version). Zum Teil umfangreich
% – Kommentar Prakāśīkā  (zur 10-Kapitel-Version)  meist sehr knapp gehalten.
% – Philologische Noten

—Category: Concordance with Ten-chapter-Version    = 7.1

—–Category: Prakāśikā
dhyāna ...    (7.1)


——Category: Philological Notes
The existence of a \emph{maṅgalaśloka} here in what appears to be the middle of the text is not
going unnoted by the two commentators (\emph{śiṣṭācāraprāptaṃ maṅgalaṃ granthamadhye racayati}
[\ldots]). Since the oldest available manuscript from Vārāṇāsī has a only three chapters, this
might indicate that the fourth chapter was an addition. 

—– Category: Jyotsnā
prathamadvitīyatṛtīyopadeśoktānām āsanakumbhakamudrāṇāṃ phalabhūtaṃ
rājayogaṃ vivakṣuḥ svātmārāmaḥ śreyāṃsi bahuvighnānīti tatra vighnabāhulyasya sambhavāt
tannivṛttaye śivābhinnagurunamaskārātmakaṃ maṅgalam ācarati nama iti | śivāya
sukharūpeśvarābhinnāya vā | tad uktaṃ namas te nātha bhagavan śivāya gururūpiṇe iti | gurave
deśikāya | yad vā, gurave sarvāntaryāmitayā nikhilopadeṣṭre śiveśvarāya | tathā ca
pātañjalasūtram pūrveṣām api guruḥ kālenānavacchedāt [yo.sū. 1.26] iti | namaḥ prahvībhāvo'stu |

kīdṛśāya śivāya ? gurave nādabindukalātmane | kāṃsyaghaṇṭānirhrādavad anuraṇanaṃ nādaḥ | bindur
anusvārottarabhāvī dhvaniḥ | kalā nādaikadeśaḥ | tā ātmā svarūpaṃ yasya sa tathā tasmai |
nādabindukalātmanā vartamānāyety arthaḥ | tatra nādabindukalātmani śive gurau nityaṃ pratidinaṃ
parāyaṇo'vahitaḥ pumān | etena nādānusandhānaparāyaṇa ity uktam | pūrvapādena guruśivayor
abhedaś ca sūcitaḥ | añjanaṃ māyopādhis tadrahitaṃ nirañjanaṃ śuddham, padyate gamyate yogibhir
iti padaṃ brahma yāti prāpnoti | tathā ca vakṣyati, nādānusandhānasamādhibhājām [4.81] ity ādinā
||1||




athedānīṃ pravakṣyāmi samādhikramam %
  \app{
    \lem[wit={V}]{uttamam}
    \rdg[wit={D}]{lakṣaṇam}}
mṛtyughnaṃ ca sukhopāyaṃ brahmānandakaraṃ param  || 4.2 ||

—– Category: Jyotsnā
samādhikramaṃ pratijānīte—atheti | athāsanakumbhakamudrākathanānataram
idānīm asminn avasare samādhikramam pratyāhārādirūpaṃ pravakṣyāmi prakarṣeṇa vivicya vakṣyāmīty
anvayaḥ | kīdṛśaṃ samādhikramam ? uttamam śrīādināthokasapādakoṭisamādhiprakāreṣūtkṛṣṭam |
punaḥ kīdṛśam ? mṛtyuṃ kālaṃ hanti nivārayatīti mṛtyughnaṃ svecchayā dehatyāgajanakaṃ
tattvajñānodayamanonāśavāsanākṣayaiḥ sukhasya jīvanmuktisukhasyopāyaṃ prāptisādhanam | punaḥ
kīdṛśam ? brahmānandakaraṃ paraṃ brahmānandakaraṃ prārabdhakarmakṣaye sati jīvabrahmaṇor
abhedenātyantikabrahmānandaprāptirūpavidehamuktikaram | tatra nirodhasamādhinā cittasya
sasaṃskārāśeṣavṛttinirodhe śāntaghoramūḍhāvasthānivṛttau, jīvann eva hi vidvān
harṣaśokābhyāṃ vimucyate ity ādiśrutyuktanirvikārasvarūpāvasthitirūpā jīvanmuktir bhavati |
paramamuktis tu prārabdhabhogānte'ntaḥkaraṇaguṇānāṃ pratiprasavenaupādhikarūpātyantikanivṛttāv
ātyantikasvarūpāvasthānaṃ pratiprasavasiddham | vyutthānanirodhasamādhisaṃskārā manasi līyante
| mano'smitāyām asmitā mahati mahān pradhān iti cittaguṇānāṃ pratiprasavaḥ pratisargaḥ svakāraṇe
layaḥ |

nanu jīvanmuktasya vyutthāne brāhmaṇo'haṃ manuṣyo'ham ity ādi vyavahāradarśanāc cittādibhir
aupādhikabhāvajananād amlena dugdhasyeva svarūpacyutiḥ syād iti cen na | samprajñātasamādhau
anubhūtātmasaṃskārasya nirodhasaṃskārasya ca tadānīṃ sattvāt | tābhyāṃ ca vyutthānasaṃskārasya
dagdhabījakalpatvād vyutthānavyavahārasyātāttvikatvaniścayāt | atāttvikānyathābhāvasya
vikāritvāprayojakatvāt | amlena dugdhasya dadhibhāvas tu tāttvika iti dṛṣṭāntavaiṣamyāc ca |
puruṣasya tv antaḥkaraṇopādhiko'haṃ brāhmaṇa ity ādivyavahāraḥ sphaṭikasya
japākusumasaṃnidhānopādhir aruṇimeva na tāttvikaḥ | japākusumāpagame sphaṭikasya
svasvarūpasthititvad antaḥkaraṇasya sakalavṛttinirodhe svasvarūpāvasthitir acyutaiva
puruṣasya ||2||

—–Category: Prakāśikā
samādhibhedān ... (7.2)




rājayogaḥ samādhiś ca unmanī ca manonmanī  |
amaratvaṃ layas tattvaṃ śūnyāśūnyaṃ paraṃ padam  || 4.3 ||

—Category: Concordance with Ten-chapter-Version    = 8.50

—– Category: Jyotsnā
samādhiparyāyān viśeṣeṇāha  rājayoga ity ādinā ślokadvayena | spaṣṭam ||3-4||

—–Category: Prakāśikā: rājayogaprayāyān āha dvābhyām (8.50–51)

—Category: Philological Notes:
In the ten-chapter version the two verses containing synonyms of rājayoga comes
at the end of chapter 8. Despite transmitting a clearly related passage this version
has some important readings. 


amanaskaṃ tathādvaitaṃ nirālambaṃ nirañjanam  |
jīvanmuktiś ca sahajā turyā cety ekavācakāḥ  || 4.4 ||

——Category: Concordance with Ten-chapter-Version  = 8.51

salile saindhavaṃ yadvat sāmyaṃ bhajati yogataḥ  |
tathātmamanasor aikyaṃ samādhir abhidhīyate  || 4.5 ||

——Category: Concordance with Ten-chapter-Version  = 7.4

—–Category: Prakāśikā
samādhyantaram āha ambusaindhavayor iti (7.4)

—– Category: Jyotsnā
tribhiḥ samādhim āha—salila iti | yadvat yathā saindhavaṃ sindhudeśodbhavaṃ lavaṇaṃ salile jale
yogataḥ saṃyogāt sāmyaṃ salilasāmyaṃ salilaikyaṃ bhajati prāpnoti, tathā tadvad ātmā ca manaś
cātmamanasī tayor ātmamanasor aikyam ekākāratā | ātmani dhāritaṃ mana ātmākāraṃ sadātmasāmyaṃ
bhajati, tādṛśam ātmamanasor aikyaṃ samādhir abhidhīyate samādhiśabdenocyate ity arthaḥ ||5||


\end{ekdverse}
\end{ekdosis}
\end{document}